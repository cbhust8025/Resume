% !TEX program = xelatex

\documentclass{resume}
%\usepackage{zh_CN-Adobefonts_external} % Simplified Chinese Support using external fonts (./fonts/zh_CN-Adobe/)
%\usepackage{zh_CN-Adobefonts_internal} % Simplified Chinese Support using system fonts
\usepackage{xeCJK}
\setCJKmainfont[BoldFont=SimHei,ItalicFont={[stkaiti.ttf]}]{SimSun}
\begin{document}
\pagenumbering{gobble} % suppress displaying page number

\name{\textbf{程博}}

\basicInfo{
  \email{bocheng8025@qq.com} \textperiodcentered\
  \phone{(+86) 16602709398} \textperiodcentered}

\section{\faGraduationCap\ \textbf{教育背景}}
\datedsubsection{\textbf{华中科技大学}}{2015.9 -- 2018.7}
\qquad{电子信息与通信学院} \qquad\qquad\textbf{信息与通信工程(硕士)}
%\vspace*{-7pt}
\datedsubsection{\textbf{华中科技大学}}{2011.9 -- 2015.7}
\qquad{电子信息与通信学院} \qquad\qquad\textbf{电子信息工程(本科)}

\section{\faCogs\ \textbf{专业技能}}
\begin{itemize}[itemsep=0.2em]
	\item 丰富的Windows C++客户端开发经验;
	\item 能够利用各种工具解决程序出现的各种稳定性和性能问题;
	\item 了解多种编程语言并具有快速学习编程语言的能力;
	\item 熟悉各种框架,具有qt等跨平台框架开发经验;
	\item 熟练掌握数据驱动优化用户体验的能力。
\end{itemize}

\section{\faUsers\ \textbf{工作经历}}
\datedsubsection{\textbf{\Large{腾讯科技(深圳)有限公司}}}{2018.07 -- Present}
\role{PC客户端C++开发}{腾讯电脑管家}

{\textbf{\large{远程项目}}}
\vspace{0.5ex}
\par{角色:项目负责人}
\par{简介:基于管家业务体系,搭建C端个人版远程控制业务,实现远程用户数规模百倍增长,初步解决远程产品可用性问题。同时,基于C端远程能力,打造B端远程商业化产品。}
\begin{itemize}[itemsep=0.4em]
  \item 从远程框架0-1构建到跨平台QT框架,从C端产品迭代到B端SDK植入,负责整体软件方案设计和核心代码编写,协调人力、需求排期,确保项目按计划推进;
  \item 针对远程各链路、功能均构建完善的用户数据视图,并对核心数据(成功率)建立实时监控告警,基于用户体验对远程进行极致地优化
	\begin{itemize}
	\item 链接成功率提升至\textbf{97.22\%}
	\item 稳定性降低至\textbf{0.89\%}
	\item 链接耗时均值降至\textbf{4秒}内
	\end{itemize}
  \item 利用各种性能工具分析远程模块性能并进行优化,远程模块Crash率\textbf{低于}管家平均模块水平\textbf{3\textperthousand}。
\end{itemize}

\vspace{2.0ex}
{\textbf{\large{增长业务}}}
\vspace{0.5ex}
\par{角色:项目负责人}
\par{简介:腾讯电脑管家下载、安装、卸载等业务直接影响电管的用户数量。}
\begin{itemize}[itemsep=0.4em]
  \item 安装包裁减:经过多轮安装包优化,1373个文件,优化至基本包仅663个文件,减少超过一半,安装包体积由109.44MB降为57.55MB,带来近万新入每天;
  \item 针对管家业务变迁,构建全新的自动化流程,涵盖了CI/CD、版本监控、数据视图等各项工作,具有丰富的开发生产实践经验;
  \item 管家产品形态多样,优化安装、卸载框架,应对各种复杂渠道环境,带来用户持续增长。
\end{itemize}

%\datedsubsection{\textbf{移动大数据分析系统}}{2016.08 -- Present}
%\role{学校方面主要负责人}
%\par{简介:基于武汉移动方面提供的软采信令大数据,结合不同用户划分的区域,对其相关的网络性能进行分析,并搭建性能分析系统平台加以显示。}
%\begin{itemize}
%  \item 基于移动软采数据,使用SQL server进行筛选并挖掘可用信息,对用户行为进行预测,例如聚类定位、移动预测、规律统计;
%  \item 利用google map api对用户RSRP、掉线、衰落以及覆盖等各项指标进行分析,整合新型KPI指标,针对指标定位问题区域,结合在线地图输出分析结果并给出解决方案;
%  \item 基于wxpython库搭建网络性能分析演示平台,输出新型KPI指标,对用户行为和指标进行相应的分析,展示相应的分析结果,定位问题区域,给出解决方案。
%\end{itemize}

%\datedsubsection{\textbf{\LaTeX\ résumé template}}{May. 2015 -- Present}
%\role{\LaTeX, Maintainer}{Individual Projects}
%An elegant \LaTeX\ résumé template, https://github.com/billryan/resume
%\begin{itemize}
%  \item Easy to be further customized or extended
%  \item Full support for unicode characters (e.g. CJK) with \XeLaTeX\
%  \item FontAwesome 4.5.0 support
%\end{itemize}

% Reference Test
%\datedsubsection{\textbf{Paper Title\cite{zaharia2012resilient}}}{May. 2015}
%An xxx optimized for xxx\cite{verma2015large}
%\begin{itemize}
%  \item main contribution
%\end{itemize}



%\section{\faHeartO\ 奖项荣誉}
%\datedline{\textit{华中科技大学校级“优秀学生干部”}}{2013.11}
%\datedline{\textit{华中科技大学“优秀本科毕业生”}}{2015.03}

%\section{\faInfo\ 自我评价}
\section{\faHeartO\ 自我评价}
\begin{itemize}[itemsep=0.2em]
  \item 持续学习新的知识,学习能力强,并能快速适应各种环境;
  \item 不畏惧疑难杂症,积极克服并善于复盘进行沉淀;
  \item 较强的抗压能力、自我驱动力和积极思考的能力。
\end{itemize}

%\section{\faInfo\ Miscellaneous}
%\begin{itemize}[parsep=0.5ex]
%  \item Blog: http://your.blog.me
%  \item GitHub: https://github.com/username
%  \item Languages: English - Fluent, Mandarin - Native speaker
%\end{itemize}

%% Reference
%\newpage
%\bibliographystyle{IEEETran}
%\bibliography{mycite}
\end{document}
