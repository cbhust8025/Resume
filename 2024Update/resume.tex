% !TEX program = xelatex

\documentclass{resume}
%\usepackage{zh_CN-Adobefonts_external} % Simplified Chinese Support using external fonts (./fonts/zh_CN-Adobe/)
%\usepackage{zh_CN-Adobefonts_internal} % Simplified Chinese Support using system fonts
\usepackage{xeCJK}
\setCJKmainfont[BoldFont=SimHei,ItalicFont={[stkaiti.ttf]}]{SimSun}
\begin{document}
\pagenumbering{gobble} % suppress displaying page number

\name{\textbf{程博}}

\basicInfo{
  \email{bocheng8025@qq.com} \textperiodcentered\
  \phone{(+86) 16602709398} }

\section{\faGraduationCap\ \textbf{教育背景}}
\datedsubsection{\textbf{华中科技大学}}{2015.9 -- 2018.7}
\qquad{电子信息与通信学院} \qquad\qquad\textbf{信息与通信工程(硕士)}
\vspace*{-7pt}
\datedsubsection{\textbf{华中科技大学}}{2011.9 -- 2015.7}
\qquad{电子信息与通信学院} \qquad\qquad\textbf{电子信息工程(本科)}

\section{\faBalanceScale\ \textbf{专业技能}}
\begin{itemize}[itemsep=0.2em]
	\item 丰富的Windows C++客户端开发经验,熟练掌握多种编程语言。
	\item 能够利用各种工具解决程序出现的稳定性和性能问题。
	\item 熟悉各种框架,具有QT等跨平台框架开发经验。
	\item 熟练掌握数据驱动优化用户体验的能力,具有快速学习编程语言的能力。
\end{itemize}

\section{\faGg\ \textbf{工作经历}}
\datedsubsection{\textbf{\Large{腾讯科技(深圳)有限公司}}}{2018.07 -- Present}
\role{PC客户端C++开发}{腾讯电脑管家}
\par{\textbf{介绍}:在CSIG安全产品部门的工作期间,积累了丰厚的Windows C++开发经验,广泛参与了腾讯电脑管家IOA云盾、小团队版、15版本、16版本及远程管家等多个C/B端产品的上线工作,深入理解了电脑管家基本框架。主要负责管家安装、体检、远程等多个核心功能模块的开发与维护,并专注于提升管家公共基础能力。此外,积极参与团队在研效、合规、人才梯队等方面的建设,确保项目顺利进行和团队高效运作。作为CSIG客户端开发通道助理,负责组织BG内部客户端开发晋级评审工作,在技术层面取得显著成果的同时,展现团队协作和组织管理方面的较强能力。}
\vspace{1.0ex}

{\textbf{\large{远程项目}}}
\vspace{0.2ex}
\par{\textbf{角色}:项目负责人}
\vspace{0.3ex}
\par{\textbf{简介}:在WebRTC开源框架的基础上,紧密围绕管家业务体系,构建C端远程控制业务,实现了远程产品从0到1的突破。进一步地,结合腾讯手机管家,拓展并丰富了PC远程能力,推出了诸如文件传输、照片搬家等实用功能。这些努力为后续B端产品(电脑管家小团队版、远程管家)的商业化奠定了坚实基础,展现了团队在远程控制领域的持续创新和专业实力。}
\begin{itemize}[itemsep=0.1em]
  \item 负责远程业务中核心业务代码的编写,利用自身对电脑管家基础框架的充分了解,将远程业务更好的融入管家现有体系中。
  \item 负责远程业务框架的整体方案设计,涵盖安全性、可扩展性、健壮性和稳定性。
  \item 负责远程业务整体数据视图的构建,实时监控及时规避风险,核心指标建设优化用户体验。
  \item 负责对接部门内外产品团队,推动管家远程能力在腾讯云iOA、腾讯内网IOA、远程管家等B端产品中的落地工作。
  \item 负责远程业务的整体性能指标的优化,涵盖链接耗时、Crash率、CPU/IO、内存占用等方面。
\end{itemize}

\vspace{0.5ex}
{\textbf{\large{公共基础}}}
\vspace{0.2ex}
\par{\textbf{角色}:核心开发}
\vspace{0.3ex}
\par{\textbf{简介}:电脑管家产品DAU达千万级别,任何微小的缺陷都可能在庞大用户群中放大,严重损害用户体验。因此,强化公共基础能力至关重要。我们需持续优化和完善这些基础能力,确保产品在高负载、高并发环境中稳定高效运行,为用户带来卓越体验。}
\begin{itemize}[itemsep=0.1em]
  \item 负责电脑管家公共基础能力建设,涵盖了安装包裁剪、下载优化、性能监控体系、Guid重复率优化等方面。
  \item 负责解决核心功能模块的用户反馈问题,利用各种工具来协助用户组高效解决用户反馈问题。
  \item 负责电脑管家研效、数据等方面建设,升级原有Jenkins到内源Coding平台,升级原有数据平台到灯塔平台。
\end{itemize}

%\datedsubsection{\textbf{移动大数据分析系统}}{2016.08 -- Present}
%\role{学校方面主要负责人}
%\par{简介:基于武汉移动方面提供的软采信令大数据,结合不同用户划分的区域,对其相关的网络性能进行分析,并搭建性能分析系统平台加以显示。}
%\begin{itemize}
%  \item 基于移动软采数据,使用SQL server进行筛选并挖掘可用信息,对用户行为进行预测,例如聚类定位、移动预测、规律统计;
%  \item 利用google map api对用户RSRP、掉线、衰落以及覆盖等各项指标进行分析,整合新型KPI指标,针对指标定位问题区域,结合在线地图输出分析结果并给出解决方案;
%  \item 基于wxpython库搭建网络性能分析演示平台,输出新型KPI指标,对用户行为和指标进行相应的分析,展示相应的分析结果,定位问题区域,给出解决方案。
%\end{itemize}

%\datedsubsection{\textbf{\LaTeX\ résumé template}}{May. 2015 -- Present}
%\role{\LaTeX, Maintainer}{Individual Projects}
%An elegant \LaTeX\ résumé template, https://github.com/billryan/resume
%\begin{itemize}
%  \item Easy to be further customized or extended
%  \item Full support for unicode characters (e.g. CJK) with \XeLaTeX\
%  \item FontAwesome 4.5.0 support
%\end{itemize}

% Reference Test
%\datedsubsection{\textbf{Paper Title\cite{zaharia2012resilient}}}{May. 2015}
%An xxx optimized for xxx\cite{verma2015large}
%\begin{itemize}
%  \item main contribution
%\end{itemize}



%\section{\faHeartO\ 奖项荣誉}
%\datedline{\textit{华中科技大学校级“优秀学生干部”}}{2013.11}
%\datedline{\textit{华中科技大学“优秀本科毕业生”}}{2015.03}

%\section{\faInfo\ 自我评价}

%\section{\faBalanceScale\ 自我评价}
%\begin{itemize}[itemsep=0.2em]
%  \item 持续学习新的知识,学习能力强,并能快速适应各种环境。
%  \item 不畏惧疑难杂症,积极克服并善于复盘进行沉淀。
%  \item 较强的抗压能力、自我驱动力和积极思考的能力。
%\end{itemize}

%\section{\faInfo\ Miscellaneous}
%\begin{itemize}[parsep=0.5ex]
%  \item Blog: http://your.blog.me
%  \item GitHub: https://github.com/username
%  \item Languages: English - Fluent, Mandarin - Native speaker
%\end{itemize}

%% Reference
%\newpage
%\bibliographystyle{IEEETran}
%\bibliography{mycite}
\end{document}
