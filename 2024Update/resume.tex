% !TEX program = xelatex

\documentclass{resume}
%\usepackage{zh_CN-Adobefonts_external} % Simplified Chinese Support using external fonts (./fonts/zh_CN-Adobe/)
%\usepackage{zh_CN-Adobefonts_internal} % Simplified Chinese Support using system fonts
\usepackage{xeCJK}
\setCJKmainfont[BoldFont=SimHei,ItalicFont={[stkaiti.ttf]}]{SimSun}
\begin{document}
\pagenumbering{gobble} % suppress displaying page number

\name{\textbf{程博}}

\basicInfo{
  \email{bocheng8025@gmail.com} \textperiodcentered\
  \phone{(+86) 13545371816} \textperiodcentered}

\section{\faGraduationCap\ \textbf{教育背景}}
\datedsubsection{\textbf{华中科技大学}}{2015.9 -- 2018.7}
\qquad{电子信息与通信学院} \qquad\qquad\textbf{信息与通信工程(硕士)}
%\vspace*{-7pt}
\datedsubsection{\textbf{华中科技大学}}{2011.9 -- 2015.7}
\qquad{电子信息与通信学院} \qquad\qquad\textbf{电子信息工程(本科)}

\section{\faUsers\ \textbf{项目经验}}
\datedsubsection{\textbf{地铁风险施工预警}}{2016.11 -- Present}
\role{负责算法实现}
\par{简介:基于地铁的监控视频,利用机器学习中深度学习的相关工具,识别施工工人的不安全行为,实时预警,保障安全。}
\begin{itemize}
  \item 针对地铁施工视频,基于wxpython库、opencv库设计相应的视频处理以及图片标记程序;
  \item 基于地铁施工的视频,基于VOC公共数据集模板进行设计数据集,利用Faster-RCNN进行多目标检测,检测AP达到0.7以上,平均耗时0.3s每张图片;
  \item 基于地铁人员的视频监控,利用FCN全卷积网络进行图像分割,给出密度图。
\end{itemize}

\datedsubsection{\textbf{移动大数据分析系统}}{2016.08 -- Present}
\role{学校方面主要负责人}
\par{简介:基于武汉移动方面提供的软采信令大数据,结合不同用户划分的区域,对其相关的网络性能进行分析,并搭建性能分析系统平台加以显示。}
\begin{itemize}
  \item 基于移动软采数据,使用SQL server进行筛选并挖掘可用信息,对用户行为进行预测,例如聚类定位、移动预测、规律统计;
  \item 利用google map api对用户RSRP、掉线、衰落以及覆盖等各项指标进行分析,整合新型KPI指标,针对指标定位问题区域,结合在线地图输出分析结果并给出解决方案;
  \item 基于wxpython库搭建网络性能分析演示平台,输出新型KPI指标,对用户行为和指标进行相应的分析,展示相应的分析结果,定位问题区域,给出解决方案。
\end{itemize}

%\datedsubsection{\textbf{\LaTeX\ résumé template}}{May. 2015 -- Present}
%\role{\LaTeX, Maintainer}{Individual Projects}
%An elegant \LaTeX\ résumé template, https://github.com/billryan/resume
%\begin{itemize}
%  \item Easy to be further customized or extended
%  \item Full support for unicode characters (e.g. CJK) with \XeLaTeX\
%  \item FontAwesome 4.5.0 support
%\end{itemize}

% Reference Test
%\datedsubsection{\textbf{Paper Title\cite{zaharia2012resilient}}}{May. 2015}
%An xxx optimized for xxx\cite{verma2015large}
%\begin{itemize}
%  \item main contribution
%\end{itemize}

\section{\faCogs\ \textbf{综合能力}}
\begin{itemize}[parsep=0.5ex]
  \item \textbf{编程相关}:
      \begin{itemize}
        \item 熟练使用C++语言,有扎实的C++语言基础及良好的编程习惯; 熟悉面向对象开发思想;熟悉常用的数据结构、算法;
        \item 熟练使用Python语言,熟悉python各种内建模块以及部分第三方模块;
      \end{itemize}
  \item \textbf{其他}:
      \begin{itemize}
        \item 能熟练阅读opencv、caffe等各种英文官方文档,解决相关编程问题;
        \item 自学能力强,乐于接触新领域,能快速上手新内容;
        \item 熟悉SQL Server、MySQL等数据库操作,了解Oracle数据库操作。
	   \item 熟悉Linux操作系统及各项命令,能够在Linux系统上进行编程开发。
        \item 了解SVM+Hog,FRCNN等各种模型,熟悉caffe等深度学习框架。
      \end{itemize}
\end{itemize}

\section{\faHeartO\ 奖项荣誉}
\datedline{\textit{华中科技大学校级“优秀学生干部”}}{2013.11}
\datedline{\textit{华中科技大学“优秀本科毕业生”}}{2015.03}

%\section{\faInfo\ 自我评价}
%\begin{itemize}
%  \item 性格开朗热情,乐于助人,善于沟通,认真负责;
%  \item 学习勤奋刻苦,喜欢在解决问题中得到进步;
%  \item 乐于接受新的知识,并能快速适应各种环境。
%\end{itemize}

%\section{\faInfo\ Miscellaneous}
%\begin{itemize}[parsep=0.5ex]
%  \item Blog: http://your.blog.me
%  \item GitHub: https://github.com/username
%  \item Languages: English - Fluent, Mandarin - Native speaker
%\end{itemize}

%% Reference
%\newpage
%\bibliographystyle{IEEETran}
%\bibliography{mycite}
\end{document}
